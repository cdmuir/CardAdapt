\documentclass[11pt, oneside]{article}

%
% Packages
%

\usepackage{geometry}
\geometry{letterpaper}
\usepackage[parfill]{parskip}    			% Activate to begin paragraphs with an empty line rather than an indent
\usepackage{graphicx}
\usepackage{booktabs}
\usepackage{topcapt}
\usepackage{amssymb}
\usepackage{amsmath}
\usepackage{natbib}

%
% Title and authors
%

\title{(working title) Local Physiological Adaptation in \textit{Mimulus cardinalis}}
\author{Christopher D. Muir and Amy L. Angert}
%\date{}							% Activate to display a given date or no date

%
% Start document
%

\begin{document}
\maketitle

\section*{Abstract}

\section*{Introduction}

% What's the big picture (this is TOO big):
It is taken for granted that genes are the foundation of evolution. If we get to the genetic basis of adaptive evolution, then we have understood the process at its deepest level. I believe this view is misguided. Genetic change is, to the best of our knowledge, the basis of evolution, but this does not imply that by understanding genetic variation we can predict phenotypic evolution. Rather, variation and constraint on phenotypic evolution are often determined by what goes on outside. Organisms must survive and reproduce in the real world of physical laws and interactions with other organisms. Physiology, the study of organismal function, tells us about the variation and constraint on ways in which organisms can meet these challenges. ...

% this paper. good start!
Local adaptation is one of the most ubiquitous observations in evolutionary ecology. In particular, adaptation to the local abiotic environment is generally thought to be the major cause of natural selection. Physiology, the study of organismal function, connects fitness to the abiotic environment, yet we actually know little about the physiological basis of adaptation to different environments within a species. To address this question, we looked at physiological variation across a broad latitudual gradient within \textit{Mimulus cardinalis}...

1. `intrinsic' physiological variation
2. variation in stress response
3. what ecological factors explain divergence (i.e. putative selective agents underlying local adaptation)

 % one paragraph general review of physiology, environment, and local adaptation

 % one paragraph review of physiology, environment, and local adaptation in Mimulus

 % one paragraph wrapping up, motivating this study

\section*{Methods}

\subsection*{Population Selection}

I chose 16 populations from throughout the range of \textit{M. cardinalis} (Table 1) that have been previously studied. Seeds were collected in the field [Amy - explain seed collection].

\begin{table}[htbp]
   \centering
   \topcaption{Table 1: 16 Focal populations}
   \begin{tabular}{@{} lllllll @{}}
      \toprule
      Name    & Region & Demo? & Pop gen? & Lat & Lon & Alt \\
      \midrule
	Hauser Creek						& South Margin		& yes	& yes	& 32.65822	& -116.53235	& 892 \\
	Cottonwood Creek					& South Margin		& yes	& no		& 32.80122	& -116.50194	& 1206 \\
	Sweetwater River / Cuyamaca Rancho	& South Margin		& yes	& yes	& 32.89928	& -116.5849	& 1223 \\
	Grade Road - to Palomar Mountain SP	& South Margin		& no		& no		& 33.31392	& -116.87129	& 1500 \\
	Whitewater Canyon					& Transverse		& yes	& yes	& 33.99329	& -116.66267	& 696 \\
	Mill Creek							& Transverse		& yes	& no		& 34.07808	& -116.87558	& 1992 \\
	West Fork Mojave River				& Transverse		& yes	& no		& 34.28425	& -117.37539	& 1087 \\
	North Fork Middle Tule				&				& yes	&		& 36.20081	& -118.65092	& 1284 \\
	Paradise Creek						& South Sierras		& yes	& yes	& 36.51776	& -118.75877	& 887 \\
	Redwood Creek					& South Sierras		& yes	& yes	& 36.69096	& -118.90961	& 1683.72 \\
	Wawona							& Central Sierras	& yes	& yes	& 37.539		& -119.654	& 1208 \\
	Rainbow Pool (RP)					& Central Sierras	& yes	& no		& 37.8188		& -120.00743	& 862.2792 \\
	Middle Yuba (Oregon Creek)			& North Sierras		& yes	& yes	& 39.39442	& -121.08302	& 425.196 \\
	Little Jameson						& North Sierras		& yes	& yes	& 39.74298	& -120.70401	& 1592 \\
	Deep Creek						& North Coast		& yes	& yes	& 41.66546	& -123.11341	& 694 \\
	Rock Creek (North Fork Umpqua)		& North Margin		& yes	& yes	& 43.37375	& -122.9575	& 311.2008 \\
	\bottomrule
   \end{tabular}
\end{table}

\subsection*{Plant propagation}

% Spring 2014 cohort
On 14 April, 2014, 3-5 seeds per family were sown directly on sand (Quikrete Play Sand, Georgia, USA) watered to field capacity in RLC4 Ray Leach cone-tainers placed in RL98 98-well trays (Stuewe \& Sons, Inc., Oregon, USA). We used pure sand both to facilitate root-washing and because \textit{M. cardinalis} typically grows in sandy, riparian soils (A. Angert, pers. obs.). Two jumbo-sized cotton balls at the bottom of cone-tainers prevented sand from washing out. Cone-tainers were continuously bottom-watered during germination by placing them in medium-sized flow trays (FLOWTMD, Stuewe \& Sons, Inc., Oregon, USA) filled part way with water, placed on benches in greenhouses at the University British Columbia campus Vancouver, Canada (LAT LON of UBC). Misters thoroughly wetted the top of the sand every two hours during the day. Most seeds germinated between 1 and 2 weeks, but we allowed 3 weeks before transferring seedlings to growth chambers. Germination was recorded daily from one to two weeks after sowing, and every few days thereafter. On 5 May (21 days after sowing), seedlings were transferred to one of two MODEL Growth Chambers (Conviron, Manitoba, Canada). We thinned seedlings to one plant per cone-tainer, leaving the center-most, largest plant. Where possible, extra seedlings were transplanted to cone-tainers that did not have any germinants. In total X\% of cone-tainers had at least one germinant. After transplanting, 743 of 768 (96.7\%) had plants that could be used in the experiment. We allowed one week at constant, non stressful conditions (20 degree day, 16 degree night) for plants to acclimate to growth chambers before starting treatments.

\subsection*{Treatments}

To do this, I will use 4 watering treatments on two populations on two soils. The watering treatments are: constant water, constant level/decreasing frequency, decreasing level/constant frequency, no water. I will monitor soil water potential in pots with no plants. Basically, I'm just trying to figure out whether one or another method is easier or more effective, and if I see any obvious response from plants in term of growth.

\subsubsection*{Temperature}

We simulated typical growing season (June and July) air temperatures at the two most thermally divergent focal sites in our study, Whitewater Canyon (High Temp) and Little Jameson (Low Temp). We downloaded daily interpolated mean, minimum, and maximum air temperature from 13 years (2000-2012) at both sites from ClimateWNA (CITE). Daily temperatures from ClimateWNA are usually highly correlated with the air temperature recorded from data loggers in the field at these sites (A. Angert, unpub. data). Hence, the ClimateWNA temperature profiles are likely to similar to actual thermal regimes experienced by \textit{M. cardinalis} in nature. To create realistic temperature regimes, we calculated the mean temperature trend from June to July using LOESS \citep{Cleveland_etal_1992}. The residuals were highly autocorrelated at both sites (warmer than average days are typically followed by more warm days) and there was strong correlation ($r = 0.65$) between sites (warm days in WWC were also warm in LIJ). The `VARselect' function in the vars package for R \citep{Pfaff_2008} indicated that a lag two Vector Autoregression (VAR(2)) model best captured the within-site autocorrelation as well as between-site correlation in residuals. We fit and simulated from the VAR(2) using the package dse \citep{Gilbert_2006?} in R. Simulated data closely resembled the autocorrelation and between-site correlation of the actual data. From simulated mean temperature, we next selected minimum and maximum daily temperatures. Mean, min, and max temperature were highly correlated at both sites. We chose min and max temperatures using site-specific fitted linear models between mean, max, and min temperature, with additional variation given by normally-distributed random deviates with variance equal to the residual variance of the linear models. For each day, the nighttime (22:00 - 6:00) chamber temperature was set to the simulated minimum temperature. During the middle of the day, chamber temperature was set to the simulated maximum temperature, with a variable period of transition between min and max so that the average temperature was equal the simulated mean temperature.

\subsubsection*{Water}

			Wet: daily/constant irrigation with cooled water
			Dry: start at week \#, gradually decrease watering frequency/level (see Nancy Emery's protocols)

			Are there any data from California about soil water content through the season?

\subsection*{Monitoring environment}

		Cross check with Poorter paper \\
		Light (PAR sensors), temp/humidity (HOBOs), soil moisture

\subsection*{Traits}

\subsubsection*{Structure}
\subsubsection*{Function}
\subsubsection*{Performance}

Growth rate: leaf expansion rate AND stem elongation rate

\subsubsection*{Floral}

\subsection*{Intrinsic variation}
	% population intercept?
\subsection*{Variation in stress response} % was high temp really stressful?
	% growth rate
	% photosynthetic
	% trait plasticity
	% variation in mortality (do I trust having multiple data points from growth measurements?)

\subsection*{Environmental correlates}
	% latitude
	% temperature (mean annual, July/june)
	% stream characterisitics - what data does Amy have from surveys?

	% approaches: direct correlation, bedazzle?

\end{document}

Vegetative: leaf symmetry or other leaf shape parameters that probably reflect developmental constraint but not affect function \\
Floral: Size, shape, herkogamy. Decide with Megan \\

	\subsection{Structure}

	Leaf traits method: \\
	Day 1: \\
	- select focal leaf (e.g. second or third leaf); \\
	- place in H20, dark, cool overnight. \\
	Day 2: \\
	- wet weight; \\
	- scan; \\
	- place in drying oven. \\

		\begin{enumerate}
			\item{Stomatal density, guard cell length, ratio from leaf opposite focal leaf (above}
			\item{Leaf size/shape (from scans)}
			\item{Leaf thickness/SLA}
			\item{Colorimetry (from scans, spec)}
			\item{stem anthocyanin (qualitative)}
		\end{enumerate}

	OPEN QUESTIONS: \\
	save wet samples in preservative (e.g. for leaf venation?) \\
	chlorophyll content? \\
	branching architecture (anything in plant protocols?)

	\subsection{Function}

		\subsubsection{during experiment}

		\begin{enumerate}
			\item{Instantaneous photosynthetic rate, $g_\text{s}$, water-use efficiency.}
				\subitem{3 min per plant per LICOR}
				\subitem{Do as many plants as possible in 1-2 weeks}
				\subitem{Before and after drought?}
			\item{Respiration}
			\item{Whole plant water-use (gravimetric)}
				\subitem{}
			\item{Leaf temperature (subset)}
			\item{Temperature response curve (subset)}
				\subitem{see protocol here: http://prometheuswiki.publish.csiro.au/tiki-index.php?page=Temperature+response+of+photosynthesis+using+a+Li-Cor+6400}
			\item{Light response curve}
				\subitem{Sampling? Only in cool, wet?}
				\subitem{At what temperature?}

		\end{enumerate}

		\subsubsection{after experiment}

		\begin{enumerate}
			\item{CN content}
			\item{carbon isotope discrimination}
		\end{enumerate}


	\subsection{Development time and Life history}

	\begin{enumerate}
		\item{Flowering time: census daily}
		\item{Starch accumulation: collect (subset) of plants}
		\item{Rhizome formation: collect on subset of plants harvested for dry mass}
	\end{enumerate}

	\subsection{Floral (2nd/3rd/etc flower?)}

	\begin{enumerate}
		\item{Flower size}
		\item{Herkogamy}
	\end{enumerate}

\section{Performance}

	Possibilities: \\
	1. (sub-)weekly census of plant height. $\approx$ 0.5 min per plant, so allow 1 day \\

	$$RGR_\text{height} = \frac{\text{ln} H_2 - \text{ln} H_1}{t_2 - t_1}$$

	2. early development (pre-treatment) leaf-area growth with camera \\
	3. harvest dry biomass before and after treatments (get whole plant leaf area using scanner at same time to ground truth camera)

	$$RGR_\text{mass} = \frac{\text{ln} M_2 - \text{ln} M_1}{t_2 - t_1}$$

\section{Determining ecological gradients that matter}

Performance tradeoffs
Isolation by ecology

\section{Tests of divergent selection}

	\subsection{Trait-environment correlation}

	We will account for relatedness by using genetic distance matrix in linear models.

	\subsection{$P_\text{ST}$ -- $F_\text{ST}$}

	Contrast trait classes with Fst distribution

	\subsection{Isolation by ecology}

	Is \textbf{G} better predicted by \textbf{E} than geographic distance? Use Gideon's method?

	\subsection{Performance tradeoffs between treatments}


\section{Ecological strategy, physiological mechanism, and tradeoff}

	\subsection{Drought escape}
		Pattern: \\
			Flowering time shorter in drier/S populations \\
			More between population variation in flowering time more variable than predicted by Fst \\
			... \\

	\subsection{Drought avoidance}
		Pattern: \\
			Maybe this should really be dehydration avoidance? \\
			Populations from drier/S avoid dehydration by limiting water loss; more variable than predicted by Fst \\

			Mechanism \\
			closing stomata earlier during soil drying

		Mechanisms: \\
			lower water-use \\

	\subsection{Drought tolerance}
		Pattern: \\
			drier/S populations have greater tolerance to low tissue water content

		Mechanisms: \\
			greater WUE \\
			ability to photosynthesize during drought \\

\section{Extension: incipient speciation as a byproduct or divergence?}

Collect pollen to measure viability.


\end{document}
